\section{Diskussion}
In diesem Versuch sollte mittels einer Langzeitmessung mit einem Szintillationsdetektor die Lebensdauer von Myonen bestimmt werden. Der bestimmte Wert für die Lebensdauer von Myonen lautet $\tau = \SI{2.10 \pm 0.023}{\micro\second}$. Dieser weicht um $\approx \SI{4.4}{\percent}$ vom Literaturwert $\tau = \SI{2.1969811 \pm 0.0000022}{\micro\second}$\cite{ParticleDataGroup:2020ssz} ab. Auch wenn die Lebensdauer nicht exakt bestimmt wurde und der Literaturwert nicht im $1\sigma$-Intervall liegt, konnte in dem Versuch gezeigt werden, dass die verwendete Apparatur zur Bestimmung der Lebensdauer von Myonen verwendet werden kann. Ein Grund für die kleinere Lebensdauer ist ein zusätzlicher Zerfallskanal negativ geladener Myonen in Materie, bei dem ein myonisches Atom entsteht.\\
Die Vereinbarkeit der bestimmten Lebensdauer mit dem Literaturwert lässt außerdem darauf schließen, dass die bestimmte Untergrundrate $N_\mathrm{Untergrund,norm} = \num{0.1004 \pm 0.00011} \frac{1}{\mathrm{Channel}}$ die tatsächliche Untergrundrate gut approximiert.\\
Die bestimmte Halbwertsbreite der Verzögerungskurve $\Delta t_\mathrm{HWB} = \SI{11.5 \pm 0.7}{\nano\second}$ ist wesentlich kleiner als der theoretisch bestimmte Wert $t_\mathrm{HWB, theo} = \SI{20}{\nano\second}$. Dies ist wünschenswert, da die Koinzidenzschaltung somit zufällige Koinzidenzen sehr gut filtern kann.
Die Zahl der gemessenen Stoppsignale beträgt $N_\mathrm{Stopp} = \num{17157}$, im VKA wurden jedoch nur $N_\mathrm{VKA} = \num{14571}$ Signale registriert. Dies lässt sich vermutlich darauf zurückführen, dass zwei fast zeitgleich eintreffende Signale vom TAC oder VKA als ein Signal wahrgenommen werden.
