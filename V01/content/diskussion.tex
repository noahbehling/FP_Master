\section{Diskussion}
In diesem Versuch sollte mittels einer Langzeitmessung mit einem Szintillationsdetektor die Lebensdauer von Myonen bestimmt werden. Der bestimmte Wert für die Lebensdauer von Myonen lautet $\tau = \SI{2.09 \pm 0.033}{\micro\second}$. Dieser weicht um $\approx \SI{4.9}{\percent}$ vom Literaturwert $\tau = \SI{2.1969811 \pm 0.0000022}{\micro\second}$\cite{ParticleDataGroup:2020ssz}. Auch wenn die Lebensdauer nicht exakt bestimmt wurde und der Literaturwert nicht im $1\sigma$-Intervall liegt, konnte in dem Versuch gezeigt werden, dass die verwendete Apparatur zur Bestimmung der Lebensdauer von Myonen verwendet werden kann. Ein Grund für die kleinere Lebensdauer ist ein zusätzlicher Zerfallskanal negativer Myonen in Materie, bei dem ein myonisches Atom entsteht.\\
Die im Fit bestimmte Untergrundrate $N_\mathrm{Untergrund} &= \num{0.11 \pm 0.028} \frac{1}{\mathrm{Channel}}$ ist ungefähr $\SI{10}{\percent}$ größer, als der zuvor über die Wahrscheinlichkeit, dass ein zweites Myon während der Suchzeit in den Szintillatortank eindringt, bestimmte Wert $N_\mathrm{Untergrund,norm} = \num{0.1004 \pm 0.00011} \frac{1}{\mathrm{Channel}}$. Dies lässt sich damit erklären, dass noch andere Einflüsse, die nicht mit einbezogen wurden, vereinzelt zu falschen Signalen führen können.\\
Die Zahl der gemessenen Stoppsignale beträgt $N_\mathrm{Stopp} = \num{17157}$, im VKA wurden jedoch nur $N_\mathrm{VKA} = \num{14571}$ Signale registriert. Dies lässt sich vermutlich darauf zurückführen, dass zwei fast zeitgleich eintreffende Signale vom TAC oder VKA als ein Signal wahrgenommen werden.
