\section{Anhang}
\label{sec:anhang}
\subsection{Aufgenommene Messwerte}

In diesem Abschnitt sind die zur Auswertung aufgenommenen Messwerte tabellarisch aufgeführt.

\begin{table}[H]
  \centering
  \caption{Zahl der gemessenen Signale in Abhängigkeit der Verzögerung zur Justage der Koinzidenzapparatur.}
  \label{tab:verz}
  \pgfplotstabletypesetfile[
  columns = {0, 1, 0, 1},
  display columns/0/.style = {column name = $\Delta t \, \si{\nano\second}$, select equal part entry of={0}{2}},
  display columns/1/.style = {column name = Counts, select equal part entry of={0}{2}},
  display columns/2/.style = {column name = $\Delta t \, \si{\nano\second}$, select equal part entry of={1}{2}},
  display columns/3/.style = {column name = Counts, select equal part entry of={1}{2}}
  ]{data/verzoegerung_comma.txt}
\end{table}

\begin{table}[H]
  \centering
  \caption{Kanalnummer in Abhängigkeit des Abstands zwischen zwei Impulsen. Bei mehreren aktiven Kanälen außerdem Zahl der Ereignisse je Kanal.}
  \label{tab:puls}
  \pgfplotstabletypesetfile[
  display columns/0/.style = {column name = $\Delta t_\mathrm{Puls}$},
  display columns/1/.style = {column name = Kanal 1},
  display columns/2/.style = {column name = Counts (Kanal 1)},
  display columns/3/.style = {column name = Kanal 2},
  display columns/4/.style = {column name = Counts (Kanal 2)},
  display columns/5/.style = {column name = Kanal 3},
  display columns/6/.style = {column name = Counts (Kanal 3)},
  ]{data/puls_comma.txt}
\end{table}

% \begin{table}[H]
%   \centering
%   \caption{Zahl der gemessen Doppelsignale im Szintillatortank in den einzelnen Kanälen.}
%   \label{tab:data}
%   \pgfplotstabletypesetfile[
%   display columns/0/.style = {column name = $\Delta t \, \si{\nano\second}$},
%   display columns/1/.style = {column name = Counts}
%   ]{channel_daten.txt}
% \end{table}

  \pgfplotstabletypesetfile[
  begin table=\begin{longtable},
    end table=\end{longtable},
  every head row/.append style = {before row = \caption{Zahl der gemessen Doppelsignale im Szintillatortank in den einzelnen Kanälen.}\label{tab:data} \\ \toprule, after row = \midrule},
  % every first row/.append style = {before row={\multicolumn{2}{c}{}\\ \caption[]{Zahl der gemessen Doppelsignale im Szintillatortank in den einzelnen Kanälen.}\\\toprule},after row=\midrule\endhead},
  columns = {0, 1, 0, 1, 0, 1},
  display columns/0/.style = {column name = Kanalnummer, select equal part entry of={0}{3}},
  display columns/1/.style = {column name = Counts, select equal part entry of={0}{3}},
  display columns/2/.style = {column name = Kanalnummer, select equal part entry of={1}{3}},
  display columns/3/.style = {column name = Counts, select equal part entry of={1}{3}},
  display columns/4/.style = {column name = Kanalnummer, select equal part entry of={2}{3}},
  display columns/5/.style = {column name = Counts, select equal part entry of={2}{3}},
  ]
  {data/channel_daten.txt}
