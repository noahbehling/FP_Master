\section{Theorie}
Das Standard Modell der Elementarteilchen unterscheidet zwischen Quarks, Leptonen und Eichbosonen.
Sowohl Quarks als auch Leptonen lassen sich in drei Generationen einteilen.
Das Myon beziehungsweise das Anti-Myon
bilden dabei die zweite Generation Leptonen mit zugehörigen Neutrinos. Sie besitzen wie das Elektron 
einen Spin $1/2$ und eine negative (beziehungsweise positive) Elementarladung. Beide Leptonen unterscheiden sich allerdings in ihrer Masse 
und in ihrer Lebensdauer. Während das Elektron stabil ist, haben Myonen eine endliche Lebensdauer, welche in diesem 
Experiment bestimmt werden soll. Des Weiteren weisen Myonen die ungefähr 206-fache Masse von Elektronen auf.
Im Folgenden wird der Begriff Myonen sowohl für Myonen als auch Anti-Myonen analog verwendent, da die hier 
relevanten Eigenschaften für beide Teilchen gleich sind.

\subsection{Enstehung kosmischer Myonen und Detektion}
Hochenergetische kosmische Strahlung erzeugt bei Auftreffen auf die Atmosphäre in ca. $10-20\;$km Höhe
Teilchenschauer. Besonders auftreffende Protonen hadronisieren dabei vor allem zu Pionen und Kaonen.
Da diese Teilchen instabil sind zerfallen sie zu Myonen (und entsprechenden Neutrinos) über
\begin{align*}
    \pi^{+} &\to \mu^{+} + \nu_{\mu} \\
    \pi^{-} &\to \mu^{-} + \bar{\nu}_{\mu}.
\end{align*}
Der Zerfallskanal des Myons lautet
\begin{align*}
    \mu^{-} &\to e^{-} + \bar{\nu}_{e} \\
    \mu^{+} &\to e^{+} + \nu_{e}.
\end{align*}
Myonen haben eine längere Lebensdauer als Pionen und können damit die Erdoberfläche erreichen.
Dort lassen sich diese mit Hilfe eines Szintilators erfassen. Im Szintilatormaterial regen die eintreffenden
Myonen die Moleküle an, welche beim wieder Abregen ein Photon abstrahlen. Dieses Photon kann dann wiederrum 
durch einen Sekundärelektronenvervielfacher detektiert werden. Dabei absorbiert dieser das Photon und emittiert
ein Elektron welches durch eine Beschleunigungsspannung und Halbleiterelementen um einen Faktor $10^5$ verstärkt wird
und damit ein Signal abgibt.
Wenn die eintreffenden Myonen dann auch im Szintilator zerfallen lösen die Elektronen beziehungsweise Positronen 
ein weiteres Signal im Szintilator aus. Durch den zeitlichen Abstand der beiden Signale lässt sich 
die Lebendauer des jeweiligen, detektierten Myons bestimmen.

\subsection{Definition der mittleren Lebendauer}
Der Zerfall instabiler Teilchen ist ein statistischer Prozess. Das heißt, dass jedes zerfallende 
Teilchen eine individuelle Lebensdauer annehmen kann, weshalb, um eine charakteristische Größe zu definieren, 
über alle Lebensdauern gemittelt werden muss. Infinitesimal betrachtet ist die Wahrscheinlichkeit $\symup{d}W$
des Zerfalls proportional zur Messzeit $\symup{d}t$ über 
\begin{equation*}
    \symup{d}W = \lambda \; \symup{d}t,
\end{equation*}
wobei $\lambda$ die Zerfallskonstante ist.
Da Teilchen unabhängig voneinander zerfallen und das Alter des Teilchens keinen Einfluss auf die Zerfallswahrscheinlichkeit 
hat, lässt sich für die Zahl der im Zeitraum $\symup{d}t$ zerfallenen Teilchen $\symup{d}t$ folgern
\begin{equation*}
    \symup{d} N = - N \; \symup{d}W = - \lambda N \; \symup{d}t.
\end{equation*}
Für sehr große Zahlen $N$ lässt sich der Zusammenhang zu 
\begin{equation*}
    \frac{N(t)}{N_0} = e^{-\lambda t}
\end{equation*}
integrieren, wobei $N_0$ die Gesamtzahl der betrachteten Teilchen ist.
Durch Bildung der Verteilungsfunktion der Lebensdauer $t$ ergibt sich 
\begin{equation*}
    \frac{\symup{d}N(t)}{N_0} = \lambda e^{-\lambda t} \; \symup{d}t.
\end{equation*}
Die charakteristische Lebensdauer $\tau$ ergibt sich als Mittwelwert aller möglichen Lebensdauern über 
\begin{equation*}
    \tau = \int_0^{\infty} \lambda t e^{-\lambda t} \; \symup{d} t = \frac{1}{\lambda},
\end{equation*}
also reziprok zur charakteristischen Zerfallskonstanten.