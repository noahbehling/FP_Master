\section{Diskussion}
Im vorbereitenden Experiment wird deutlich, dass das Computerprogramm eine deutlich höhere
Auflösung hat, als das Oszilloskop, also deutlich genauere Werte liefert. Gleichzeitig
bestätigt die Messung am Oszilloskop und der Vergleich die Validität des Computerprogramms.
Das akustische Modell des Wasserstoffatoms kann als gutes Analogon zum quantenmechanischen 
System betrachtet werden. Das Modell liefert Werte, die 
den theoretischen Werten im Genauigkeitsrahmen des Experimentes gut genug entsprechen um mit
anderen Effekten wie Reflexion oder Beeinflussung von außen erklärt werden zu können.
Die Aufspaltung der Zustände durch Einfügen eines Ringes, welcher dabei die $m$-Zustände aufspaltet,
liefert ebenfalls sehr gute Ergebnisse
und zeigt nochmal deutlicher, dass die Analogie hier funktioniert. Dass die Messwerte bei 
eingesetzem Ring tatsächlich noch näher an der Theorie liegen, liegt daran, dass dabei nur 
ein $l$ betrachtet wurde, während bei der Messung ohne Ring stärkere Abweichungen vorliegen.
Ein Grund hierfür kann die Abhängigkeit der Quantenzahl $l$ von der Frequenz sein. Bei höheren
Frequenzen lässt sich diese genauer bestimmen.\\
Bei der Untersuchung des Wasserstoffmoleküls ließen sich die verschiedenen Resonanzpeaks,
zumindest bei größeren Blendendurchmessern gut voneinander trennen. Bei kleineren Durchmessern
ist die schwieriger oder auch unmöglich, was aber zu erwarten war. 
Die Nähe dieser Peaks zueiander ist auch in der Phasendifferenz zwischen oberer und unterer
Kugel zu erkennen, die bei einem Peak nicht 180$°$ entspricht. Die Verteilung der Peaks 
in Abhängigkeit des Azimutwinkels ist ebenfalls nur für den dritten Peak, welcher frei von den anderen steht,
gut erkennbar. Die anderen beiden Peaks weisen eine Winkelverteilung ohne erkennbares Muster
auf. Ein Grund könnte hierfür die Überlagerung (und damit schwere Unterscheidbarkeit im Spektrum)
der Zustände $2 \sigma_{\text{g}}$ und $1 \pi_{\text{u}}$ sein. Der dritte Peak ist aber gut erkennbar und kann 
dem $2 \sigma_{\text{u}}$ Zustand gut zugeordnet werden. \\
Die Modellierung des eindimensionalen Festkörpers lief ebenfalls zufriedenstellend ab, sodass 
direkte Analogien zu Bändern und Bandlücken im quantenmechanischen Modell gefunden werden konnten.

Insgesamt verlief der Versuch gut; eine höhere Genauigkeit ließe sich vor allem durch 
kleinschrittigere Frequenzspektrenaufnahmen und längere Messzeiten pro Schritt erreichen.
