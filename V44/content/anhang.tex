\section{Anhang}
\label{sec:anhang}
\begin{lstlisting}
import numpy as np
import matplotlib.pyplot as plt


# Reflektivitätsscan:
angle_refl, intensity_refl = np.genfromtxt("data/scan1.UXD", unpack=True)
# Diffuser Scan
angle_diff, intensity_diff = np.genfromtxt("data/scan2.UXD", unpack=True)

# Reflektivität: R=I/5I_0; 5 durch 5s Messzeit
I_0 = 9.7272e05  # Werte jeweils aus vorherigen Berechnungen
R_refl = intensity_refl / (5 * I_0)
R_diff = intensity_diff / (5 * I_0)
R = R_refl - R_diff

# Geometriefaktor anwenden; für Vergleich mit Parratt
a_g = 0.6875658564108561
R_G = np.zeros(np.size(R))
for i in np.arange(np.size(a)):
    if(a[i] <= a_g and a[i] >0 ):
        R_G[i] = R[i] * np.sin(np.deg2rad(a_g)) / np.sin(np.deg2rad(a[i])) 
    else:
        R_G[i] = R[i]

# Parratt-Algorithmus

# a_i Einfallswinkel, Brechuingsindex: n1 Luft, n2 Polystyrol, n3 Silicium
# Rauigkeiten: sigma1 Polystyrol, sigma2 Silizium
# Schichtdicken: d1=0 (Luft), d2: Dicke der Schicht Polystyrol

n1 = 1.0
d1 = 0.0
wellenlaenge = 1.54e-10
# Ermittelte Werte durch Anpassung des Parratt
delta2 = 0.6 * 10 ** (-6)
delta3 = 6.0 * 10 ** (-6)
sigma1 = 5.5 * 10 ** (-10)  # m
sigma2 = 6.45 * 10 ** (-10)  # m
d2 = 8.6 * 10 ** (-8)  # m
b2 = (delta2 / 40) * 1j
b3 = (delta3 / 200) * 1j


def parratt(a_i, delta2, delta3, sigma1, sigma2, d2, b1, b2):
    n2 = 1.0 - delta2 + b2 
    n3 = 1.0 - delta3 + b3
    a_i = np.deg2rad(a_i)
    k = 2 * np.pi / wellenlaenge
    kd1 = k * np.sqrt(n1 ** 2 - np.cos(a_i) ** 2)
    kd2 = k * np.sqrt(n2 ** 2 - np.cos(a_i) ** 2)
    kd3 = k * np.sqrt(n3 ** 2 - np.cos(a_i) ** 2)

    r12 = (kd1 - kd2) / (kd1 + kd2) * np.exp(-2 * kd1 * kd2 * sigma1 ** 2)
    r23 = (kd2 - kd3) / (kd2 + kd3) * np.exp(-2 * kd2 * kd3 * sigma2 ** 2)

    x2 = np.exp(-2j * kd2 * d2) * r23
    x1 = (r12 + x2) / (1 + r12 * x2)

    return np.abs(x1) ** 2


params = [delta2, delta3, sigma1, sigma2, d2, b1, b2]
print(params)
parratt_curve = parratt(angle_refl, *params)

# Kritischer Winkel
a_Poly = np.rad2deg(np.sqrt(2 * delta2))
a_Si = np.rad2deg(np.sqrt(2 * delta3))

print("Kritischer Winkel Polystyrol: ", a_Poly)
print("Kritischer Winkel Silicium: ", a_Si)


# Plot

plt.axvline(a_Poly, linewidth=0.5, color="green", label="α Polystyrol")
plt.axvline(a_Si, linewidth=0.5, color="pink", label="α Silizium")
plt.plot(angle_refl, parratt_curve, "-", label="Parratt-Kurve")
plt.plot(angle_refl, R_G, "-", label="gem. Reflektivität mit Korrektur durch Geometriefaktor")
plt.xlabel("α / °")
plt.ylabel("R")
plt.yscale("log")
plt.grid(alpha=0.4)
plt.legend()
plt.savefig("parratt.pdf")
\end{lstlisting}