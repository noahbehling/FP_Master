\section{Diskussion}
Die durchgeführten Scans zur Justierung des Strahls verliefen gut entsprechend der Entwartung.
Der gemessene Geometriewinkel $\alpha = 0,68°$ entspricht dem theoretischen Wert von 
$\alpha_{\text{theo}} = 0,6876°$ mit einer relativen Abweichung von $1,1\%$ sehr gut.
Dies spricht daher allgemein für eine gute Justierung.
Über die gut erkennbaren Kiessig-Oszillationen wurde eine Schichtdicke zu
$d_{\text{Kiessig}} =(8,7 \pm 0,3)\cdot 10^{-8} \; \symup{m} $ bestimmt. Über den Parratt-Algortihmus 
konnte ebenfalls eine Schichtdicke bestimmt werden, welche mit 
$ d_{\text{Parratt}} = 8,6 \cdot 10^{-8} \;\symup{m}$ mit der vorherigen
Abschätzung im Bereich dessen Ungenauigkeit übereinstimmt. Die Abschätzung der Schichtdicke über die Kiessig-Oszillationen
wird also als gut angenommen. Dass es sich bei der durch den Parratt-Algorithmus erhaltene
Schichtdicke um die richtige Schichtdicke handelt, lässt sich in \autoref{fig:parratt} erkennen.
Eine Veränderung der Schichtdicke hätte eine Veränderung der Frequenz der Kiessig-Oszillationen
zur Folge, welche aber hier gut mit der gemessenen Frequenz übereinstimmt. \\
Allgemein ist aber zu Beachten, dass die Parameter des Parratt-Algorithmus durch manuelle,
menschliche Anpassung an die gemessene Kurve erfolgt ist und daher auch nur mit menschlicher 
Genauigkeit behaftet sind.\\
Über den Parratt-Algorithmus wurden des Weiteren die Dispersionen
\begin{align*}
    \delta_{\text{Poly}} &= 6 \cdot 10^{-7} \\
    \delta_{\text{Si}} &= 6 \cdot 10^{-6}
\end{align*}
bestimmt.
Die theoretischen Werte \cite{tolan} der Dispersionen sind 
\begin{align*}
    \delta_{\text{Poly}} &= 3,5 \cdot 10^{-6} \\
    \delta_{\text{Si}} &= 7,6 \cdot 10^{-6}.
\end{align*}
Für das Polystyrol ergibt sich dadurch eine relative Abweichung von $82,9 \%$ und 
für das Silizium eine relative Abweichung von $21,0\%$. Die im Vergleich zum Silizium sehr 
hohe Abweichung der Dispersion Polystyrol lässt darauf schließen, dass wahrscheinlich ein 
systematischer Fehler vorliegt. Worin genau dieser besteht, kann allerdings nicht angegeben werden.\\
Aus den bestimmten Dispersionen konnten ebenfalls die kritischen Winkel der Materialien bestimmt
werden. Die Ergebnisse lauten 
\begin{align*}
    \alpha_{\text{c, Pol}} &= 0,063°\\
    \alpha_{\text{c, Si}} &= 0,198°.
\end{align*}
Im Vergleich zu den Theoriewerten \cite{tolan} 
\begin{align*}
    \alpha_{\text{c, Pol}} &= 0,153°\\
    \alpha_{\text{c, Si}} &= 0,223°
\end{align*}
ergeben sich relative Abweichungen bei Polystyrol von $58,8\%$ beziehungsweise bei Silizium von $11,2\%$. Die Abweichung 
Die hohe Abweichung bei Polystyrol ist eine direkte Folge der hohen Unsicherheit bei der Dispersionsmessung.
Der kritische Winkel von Silizium weist ebenfalls eine sehr geringe Abweichung auf, weswegen 
die Messung des Siliziums als erfolgreich betrachtet werden kann.\\
Mit dem Parratt-Algorithmus konnte ebenfalls die Rauigkeit zwischen Polystyrol und 
Silizium zu 
\begin{equation*}
    \sigma_{\text{Poly, Si}} = 6,5 \cdot 10^{-10} \;\symup{m}
\end{equation*}
bestimmt werden. Die Parratt-Kurve ist sehr empfindlich bezüglich dieser Parameter, weswegen 
die Bestimmung der Rauigkeit im Rahmen der Genauigkeit der Anpassung des Parratt-Algortihmus
als gut angenommen werden kann.
Die zweite Rauigkeit zwischen Luft und Polystyrol 
\begin{equation*}
    \sigma_{\text{Luft, Poly}} = 5,5 \cdot 10^{-10} \;\symup{m} 
\end{equation*}
hat eine eher untergeordnete Rolle in der Anpassung gespielt, weshalb keine Aussage über
ihre Genauigkeit getroffen werden kann.\\
Allgemein ist die manuelle Anpassung der Parameter im Parratt-Algortihmus sehr ungenau 
und daher sind alle Ergebnisse mit diesem Algortihmus mit dieser Ungenauigkeit zu betrachten.
Eine andere Einstellung der Parameter hätte möglicherweise ähnliche oder sogar bessere 
Ergebnisse hervorgebracht. Die Parratt-Kurve beschreibt die gemessene Reflektivität
aber gut genau, sodass den Ergebnissen dennoch eine Aussagekraft zugeschrieben werden kann, wie 
auch die Vergleiche mit Literaturwerten zeigen. 
