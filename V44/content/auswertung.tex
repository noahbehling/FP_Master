\section{Auswertung}
Die hier durchgeführte Datenanalyse wurde mithilfe der \textit{Python} 3.8.0 Bibliothek
\textit{Numeric Python} \cite{numpy} erstellt und mit \textit{matplotlib} \cite{matplotlib} grafisch
dargestellt. Des Weiteren wird die Bibliothek \textit{uncertainties} \cite{uncertainties} genutzt,
um die Fehlerrechnung zu automatisieren.
Zunächst werden Scans, welche bei der Justierung aufgenommen wurden, ausgewertet und 
aus ihnen Konstanten bestimmt, welche für die weitere Auswertung nötig sind.
\subsection{Detektorscan}
Zunächst werden im Detektorscan die Halbwertsbreite und die maximale Intensität bestimmt.
Dazu wird eine Gaußfunktion der Form 
\begin{equation*}
    I(\alpha) = \frac{a}{\sqrt{2 \pi \sigma^2}} \cdot \exp\left(- \frac{(\alpha - \mu)^2}{2\sigma^2}\right) + b 
\end{equation*}
an die Messwerte angepasst. Eine Darstellung der Messwerte und der Ausgleichskurve ist in 
\autoref{fig:detektor} zu sehen.
\begin{figure}
    \centering
    \includegraphics[width=0.7\textwidth]{build/plot_detektorscan.pdf}
    \caption{Messwerte des Detektorscans mit Ausgleichskurve und eingezeichneter Halbwertsbreite.}
    \label{fig:detektor}
\end{figure}
Mittels der \textit{Python}-Bibliothek \textit{SciPy} \cite{scipy} ergeben sich die Parameter
der Gaußfunktion zu 
\begin{align*}
    a &= ( 1,06 \pm 0,01)\cdot 10^5 \\
    b &= ( 1,1 \pm 0,3) \cdot 10^4 \\
    \sigma &= (0.0441 \pm 0.0006)° \\
    \mu &= (-0,0017 \pm 0.0005)°.
\end{align*}
Mittels der angelegten Gaußkurve ergibt sich die Halbwertsbreite FWHM, sowie die maximale 
Intensität $I_{\text{max}}$ zu 
\begin{align*}
    I_{\text{max}} &= (9,7 \pm 0,2)\cdot 10^5 \\
    \text{FWHM} &= (0,105 \pm 0,001)°.
\end{align*}

\subsection{Z-Scan}
Aus dem ersten durchgeführten Z-Scan kann die Strahlbreite ermittelt werden. Dazu wird die 
abfallende Intensität die auftritt, wenn die Probe in den Strahl gefahren wird. Dargestellt ist dies 
in \autoref{fig:z-scan}. Die Strahlbreite entspricht dann dem geringsten Abstand zwischen
maximaler Intensität (wenn der Strahl die Probe noch nicht trifft) und minimaler Intensität
(wenn der komplette Strahl die Probe trifft).
\begin{figure}
    \centering
    \includegraphics[width=0.8\textwidth]{build/plot_zscan.pdf}
    \caption{Messwerte des ersten Z-Scans mit eingezeichneter Strahlbreite.}
    \label{fig:z-scan}
\end{figure}
Die Strahlbreite ergibt sich dadurch zu 
\begin{equation*}
    d_0 \approx 0,24\; \symup{mm}.
\end{equation*} 

\subsection{Rockingscan}
Der erste durchgeführte Rockingscan ist in \autoref{fig:Rockingscan} zu sehen. Aus diesem lässt
sich der Geometriewinkel $\alpha_g$ ablesen. 
\begin{figure}
    \centering 
    \includegraphics[width=0.8\textwidth]{build/plot_rockingscan}
    \caption{Rockingscan mit $2 \alpha = 0$ und eingezeichnete Geometriewinkel.}
    \label{fig:Rockingscan}
\end{figure}
Der gemessene Geometriewinkel ist 
\begin{equation*}
    \alpha_g = 0,68°.
\end{equation*}
Mit der Strahlbreite $d_0$ und der Probenlänge $D = 20\;\symup{mm}$ ergibt sich über 
\autoref{eqn:geofaktor} der theoretische Wert
\begin{equation*}
    \alpha_{\text{Theorie}} = 0,6876°.
\end{equation*}

\subsection{Reflektivitätsscan}
Mittels des aufgenommenen Reflektivitätsscan wird die Probe untersucht. Zunächst wird dazu der 
diffuse Scan vom Reflektivitätsscan abgezogen um Effekte durch Rückstreeung zu vermeiden und so 
die tatsächliche Reflektivität zu bestimmen. Dies ist in \autoref{fig:messung}
dargestellt. Des Weiteren ist die Reflektivität mit der Korrektur durch den über 
\autoref{eqn:geofaktor} berechneten Geometriefaktor zu sehen. Mit dieser wird auch im Folgenden 
weiter gerechnet. Zum Vergleich mit theoretischen Werten ist außerdem die Kurve der Fresnelreflektivität 
einer idealen glatten Siliziumoberfläche zu sehen. Diese wird über 
\autoref{eqn:fresnel} bestimmt, wobei der Literaturwert des kritischen Winkels $\alpha_{\text{Si}} = 0,223°$ \cite{tolan} ist.
Außerdem ist zu Beachten, dass statt der Intensität $I$ die Reflektivität $R$ verwendet wird,
mit
\begin{equation*}
    R = \frac{I}{5 \cdot I_{\text{max}}}.
\end{equation*}
\begin{figure}
    \centering
    \includegraphics[width=0.8\textwidth]{build/plot_reflektivitaet.pdf}
    \caption{Gemessene Reflektivität, Korrektur mit dem Geometriefaktor und theoretische Fresnelreflektivität von idealem, glatten Silizium aufgetragen gegen den Winkel.}
    \label{fig:messung}
\end{figure}

Um die Schichtdicke der Polystyrolschicht zunächst abzuschätzen werden die Abstände der Minima der Kiessig-Oszillationen $\Delta \alpha_i$
bestimmt und gemittelt. Dazu wird wiederum die \textit{Python}-Bibliothek \textit{SciPy} \cite{scipy}
verwendet. Die gefundenen Minima sind ebenfalls in \autoref{fig:messung} eingezeichnet.
Es ist zu Beachten, dass nur Minima im Bereich $\alpha = [0,3°; 1,0°]$, also in welchem 
die Minima noch gut findbar sind, verwendet werden.
Der gemittelte Abstand der Minima ergibt sich zu 
\begin{equation*}
    \Delta \alpha = (8,9 \pm 0,3) \cdot 10^{-4}\,°
\end{equation*}
und damit über \autoref{eqn:abstandoderso} mit der Wellenlänge der $K_{\alpha 1}$-Linie 
$\lambda = 1,541 \cdot 10^{-10} \; \symup{m}$ eine Schichtdicke von
\begin{equation*}
    (8,7 \pm 0,3)\cdot 10^{-8} \; \symup{m}.
\end{equation*}

\subsection{Bestimmung des Dispersionsprofils mit dem Parratt-Algorithmus}
Mittels des Parratt-Algorithmus \cite{parratt} lässt sich das Dispersionsprofil der Probe bestimmen.
Dazu werden modifizierte Fresnelkoeffizienten verwendet um die Rauigkeit der Probe mit 
einzubeziehen. Das hier verwendete Programm ist in \autoref{sec:anhang} zu sehen.
Es werden dabei zwei Schichten betrachtet; eine Schicht aus Polystyrol und eine Schicht 
aus Silizium.
Des Weiteren wird die Umgebungsluft als erste Schicht mit einer Schichtdicke und Dispersion 
von $\delta = d = 0$ angenommen.
Die so berechnete Reflektivität wird mit der gemessenen, korrigierten Reflektivität 
und über Parameter manuell angepasst um eine möglichst gute Übereinstimmung zu erhalten. 
Die endgültige Kurve ist in \autoref{fig:parratt} zu sehen. Die verwendeten Parameter lauten
\begin{align*}
    \delta_{\text{Poly}} &= 6 \cdot 10^{-7} \\
    \delta_{\text{Si}} &= 6 \cdot 10^{-6} \\
    \sigma_{\text{Luft, Poly}} &= 5,5 \cdot 10^{-10} \;\symup{m}  \\ 
    \sigma_{\text{Poly, Si}} &= 6,5 \cdot 10^{-10} \;\symup{m}  \\
    d_2 &= 8,6 \cdot 10^{-8} \;\symup{m}\\
    \beta_{\text{Poly}} &= \frac{\delta_{\text{Poly}}}{200}\\
    \beta_{\text{Si}} &= \frac{\delta_{\text{Si}}}{40},
\end{align*}
wobei die Parameter $\beta_{\text{Poly}}$ und $\beta_{\text{Si}}$ der Literatur\cite{parratt} entnommen sind.
\begin{figure}
    \centering
    \includegraphics[width=0.8\textwidth]{build/plot_messung.pdf}
    \caption{Darstellung der Parratt-Reflektivität im Vergleich zur gemessenen Reflektivität mit eingezeichneten kritischen Winkeln.}
    \label{fig:parratt}
\end{figure}
Des Weiteren können die kritischen Winkel von Silizium und Polystyrol über \autoref{eqn:nocheine} zu 
\begin{align*}
    \alpha_{\text{c, Pol}} &= 0,063°\\
    \alpha_{\text{c, Si}} &= 0,198°
\end{align*}
bestimmt werden. Diese Werte sind ebenfalls in \autoref{fig:parratt} dargestellt.