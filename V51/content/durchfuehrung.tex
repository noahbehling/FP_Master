\section{Durchführung}
In diesem Versuch werden die in \autoref{sec:theorie} beschriebenen Schaltungen 
aufgebaut und untersucht. Hierbei wird das Operationsverstärkermodell LM741 verwendet.
Dieser bedarf einer Betriebsspannung von $\pm 15 \; \symup{V}$. Alle Schaltungen 
werden dabei auf einer Steckplatine realisiert.

Beim invertierenden Linearverstärker wird bei einer Eingangsspannung von $U_e = 50 \; \symup{mV}$
der Verstärkungsfaktor in Abhängigkeit der Frequenz gemessen. Außerdem wird die Phasenverschiebung
zwischen Eingangssignal und Ausgangssignal aufgenommen.
Beide Messungen werden dabei für insgesamt drei Verstärkungsfaktoren durchgeführt.\\
Anschließend werden die Schaltungen für den Umkehr-Integrator und den invertierenden 
Differenzierer aufgebaut. An beiden wird der Zusammenhang zwischen Ausgangsspannung und 
Frequenz überprüft. Des Weiteren werden jeweils Oszilloskopbilder der Integration beziehungsweise
Differentiation von Sinus-, Dreick- und Rechtecksignalen aufgenommen. \\
Im nächsten Schritt wird der Schmitt-Trigger aufgebaut. An ihm werden die Schwellspannungen 
gemessen, an denen das System kippt. Dazu werden auch Oszilloskopbilder am Kipppunkt aufgenommen.\\
Durch Kombination von Schmitt-Trigger und Integrator wird ein Signalgenerator aufgebaut.
An ihm wird die Frequenz und Amplitude des erzeugten Signals vermessen. Dies geschieht wiederum 
über ein Oszilloskop. \\
Abschließend wird die Schaltung der variierenden Amplituden aufgebaut und an ihr Bilder 
am Oszilloskop für den gedämpften beziehungsweise enddämpften Fall aufgenommen werden.
Im gedämpften Fall muss dazu zusätzlich eine Rechteckspannung an den nicht-invertierenden 
Eingang des zweiten Operationsverstärkers angelegt werden.  