\section{Aufgenommene Messwerte}
\label{sec:anhang}
In diesem Abschnitt sind die aufgenommenen Messwerte zu den verschiedenen Messreihen tabellarisch aufgeführt.

\begin{table}[H]
  \centering
  \caption{Erste Messreihe zum invertierenden Linearverstärker mit $R_1 = \SI{1}{\kilo\ohm}$ und $R_2 = \SI{100}{\kilo\ohm}$. Amplitude $U_\mathrm{a}$ der Ausgangsspannung und Zeitverschiebung $\Delta t$ zwischen Ein- und Ausgangsspannung in Abhängigkeit von der Frequenz $f$ der Eingangsspannung in einem Bereich von $\SI{10}{\hertz}$ bis $\SI{1}{\mega\hertz}$.}
  \label{tab:lin_1}
  \pgfplotstabletypesetfile[
  display columns/0/.style = {column name = $f$/$\si{\kilo\hertz}$},
  display columns/1/.style = {column name = $U_\mathrm{a}$/$\si{\volt}$},
  display columns/2/.style = {column name = $\Delta t$/$\si{\micro\second}$}
  ]{data/linearverstaerker_1.txt}
\end{table}

\begin{table}[H]
  \centering
  \caption{Zweite Messreihe zum invertierenden Linearverstärker mit $R_1 = \SI{1}{\kilo\ohm}$ und $R_2 = \SI{150}{\kilo\ohm}$. Amplitude $U_\mathrm{a}$ der Ausgangsspannung und Zeitverschiebung $\Delta t$ zwischen Ein- und Ausgangsspannung in Abhängigkeit von der Frequenz $f$ der Eingangsspannung in einem Bereich von $\SI{10}{\hertz}$ bis $\SI{1}{\mega\hertz}$.}
  \label{tab:lin_2}
  \pgfplotstabletypesetfile[
  display columns/0/.style = {column name = $f$/$\si{\kilo\hertz}$},
  display columns/1/.style = {column name = $U_\mathrm{a}$/$\si{\volt}$},
  display columns/2/.style = {column name = $\Delta t$/$\si{\micro\second}$}
  ]{data/linearverstaerker_2.txt}
\end{table}

\begin{table}[H]
  \centering
  \caption{Dritte Messreihe zum invertierenden Linearverstärker mit $R_1 = \SI{15}{\kilo\ohm}$ und $R_2 = \SI{100}{\kilo\ohm}$. Amplitude $U_\mathrm{a}$ der Ausgangsspannung und Zeitverschiebung $\Delta t$ zwischen Ein- und Ausgangsspannung in Abhängigkeit von der Frequenz $f$ der Eingangsspannung in einem Bereich von $\SI{10}{\hertz}$ bis $\SI{1}{\mega\hertz}$.}
  \label{tab:lin_3}
  \pgfplotstabletypesetfile[
  display columns/0/.style = {column name = $f$/$\si{\kilo\hertz}$},
  display columns/1/.style = {column name = $U_\mathrm{a}$/$\si{\volt}$},
  display columns/2/.style = {column name = $\Delta t$/$\si{\micro\second}$}
  ]{data/linearverstaerker_3.txt}
\end{table}

\begin{table}[H]
  \centering
  \caption{Aufgenommene Messwerte zur Untersuchung des Integrators mit $R = \SI{10}{\kilo\ohm}$ und $C = \SI{100}{\nano\farad}$. Ausgangsspannung $U_\mathrm{a}$ in Abhängigkeit der Frequenz $f$ der Eingangsspannung in einem Bereich von $\SI{50}{\hertz}$ bis $\SI{5}{\kilo\hertz}$.}
  \label{tab:int}
  \pgfplotstabletypesetfile[
  display columns/0/.style = {column name = $f$/$\si{\kilo\hertz}$},
  display columns/1/.style = {column name = $U_\mathrm{a}$/$\si{\volt}$},
  ]{data/integrator.txt}
\end{table}

\begin{table}[H]
  \centering
  \caption{Aufgenommene Messwerte zur Untersuchung des Differenzierers mit $R = \SI{100}{\kilo\ohm}$ und $C = \SI{22}{\nano\farad}$. Ausgangsspannung $U_\mathrm{a}$ in Abhängigkeit von der Frequenz $f$ der Eingangsspannung im Bereich von $\SI{20}{\hertz}$ bis $\SI{2}{\kilo\hertz}$.}
  \label{tab:diff}
  \pgfplotstabletypesetfile[
  display columns/0/.style = {column name = $f$/$\si{\kilo\hertz}$},
  display columns/1/.style = {column name = $U_\mathrm{a}$/$\si{\volt}$},
  ]{data/differenzierer.txt}
\end{table}
