\section{Theorie}
\label{sec:theorie}
Mittels eines Operationsverstärkers lässt sich eine Ausgangsspannung $U_a$ erzeugen, welche 
proportional zur Differenz zweier angelegter Eingangsspannungen ist. 
Der Operationsverstärker ist ein Bauteil welches im Allgemeinen über zwei Spannungseingänge, zwei Anschlüsse 
für die Betriebsspannung $U_B$ und einen Ausgang 
verfügt. In Schaltskizzen wird 
das in \autoref{fig:opskizze} gezeigte Symbol für den Operationsverstärker verwendet.
\begin{figure}[H]
    \centering
    \includegraphics[width=0.2\textwidth]{op.png}
    \caption{Schaltzeichen des Operationsverstärkers \cite{anleitung}.}
    \label{fig:opskizze}
\end{figure}
Dabei beschreibt 
\textbf{+} die Eingangsspannung, die in Phase geschaltet ist und daher auch als nicht-invertierender
Eingang beschrieben wird, während \textbf{-} der gegenphasige oder invertierende Eingang 
zur Ausgangsspannung ist. Die Ausgangsspannung ist dadurch definiert als 
\begin{equation*}
    U_a = v \cdot  (U_+ - U_-),
\end{equation*}
wobei $U_{\pm}$ die angelegte Spannung an den jeweiligen Eingängen und $v$ die Leerlaufverstärkung ist.
Die Ausgangsspannung ist dabei durch die Betriebsspannung über 
\begin{equation*}
    - U_B < U_a < U_B
\end{equation*}
begrenzt.

Zur theoretischen Beschreibung eines Operationsverstärkers werden in der Regel idealisierende 
Annahmen getroffen. Im idealen Operationsverstärker ist sowohl die Verstärkung, als auch 
der Eingangswiderstand unendlich groß ($v = \infty$ und $R_E = \infty$),
während es keinen Ausgansgwiderstand gibt ($R_A = 0$). \\
In einem realen Operationsverstärker sind diese Idealisierungen nicht zu erreichen, sodass 
Leerlaufverstärkung und Eingangswiderstand nur möglichst groß und der Ausgansgwiderstand möglichst 
klein gehalten werden. Da die Theorie Operationsverstärker mit diesen Idealisierungen beschreibt, 
kommen in der Realität noch Korrekturterme hinzu.

\subsection{Invertierender Linearverstärker}
Die Schaltskizze eines invertierenden Linearverstärkers ist in \autoref{fig:lin} zu sehen.
\begin{figure}[H]
    \centering
    \includegraphics[width=0.4\textwidth]{linear.png}
    \caption{Schaltskizze des invertierenden Linearverstärkers \cite{anleitung}.}
    \label{fig:lin}
\end{figure}
Bei diesem Aufbau wird der invertierende Eingang über zwei Widerstände rückgekoppelt,
sodass ein Teil der 
Ausgangsspannung an den invertierenden Eingang zurückgegeben wird. An dieser Anschlussstelle 
ist die Spannung gering, sodass sich über die Kirchhoff'schen Regeln die Gleichung 
\begin{equation*}
    \frac{U_e}{R_1} + \frac{U_a}{R_2} = 0
\end{equation*}
bestimmen lässt. Die Bezeichnungen stimmen dabei mit denen aus \autoref{fig:lin} überein.
Die ideale Leerlaufverstärkung $v'$ ist damit durch 
\begin{equation}
    \label{eqn:verhaeltnis}
    v' = \frac{U_a}{U_e} = - \frac{R_2}{R_1}
\end{equation}
gegeben. Im Idealfall ist die Leerlaufverstärkung also nur durch das Verhältnis der Widerstände
gegeben, da die Spannung $U_+$ am nicht-invertierenden Eingang verschwindet.
An der Anschlussstelle (As) ergibt sich für einen unbelasteten Spannungsteiler ($I_\text{As}=0$)
\begin{equation*}
    U_\text{AS} = - \frac{U_a}{v}
\end{equation*}
mit der realen Verstärkung $V$ und damit 
\begin{equation*}
    \frac{U_\text{AS} - U_1}{U_a - U_e} = \frac{R_1}{R_1 + R_2}.
\end{equation*}   
Über \autoref{eqn:verhaeltnis} ergibt sich damit
\begin{equation*}
    \frac{1}{v'} \approx \frac{1}{v} + \frac{R_1}{R_2}.
\end{equation*}
Durch die Rückkopplung erhöht sich also die Stabilität der Verstärkungschaltung.
Die Bandbreite $B$ gibt den Frequenzbereich konstanter Verstärkung an.
Die Verstärkung $v$ und die Bandbreite $B$, hängen dabei über die Beziehung
\begin{equation*}
    v \cdot B = \text{const.}
\end{equation*}
zusammen.
\subsection{Umkehr-Integrator}
Die Schaltskizze des Umkehrintegrators ist in \autoref{fig:umkehrint} zu sehen.
\begin{figure}[H]
    \centering
    \includegraphics[width=0.4\textwidth]{integrator.png}
    \caption{Schaltskizze des Umkehr-Integrators \cite{anleitung}.}
    \label{fig:umkehrint}
\end{figure}
Im Vergleich zur vorherigen Schaltung wird der Widerstand $R_2$ durch einen Kondensator $C$
ersetzt.
Dadurch wird die Eingangsspannung $U_e$ über die Zeit integriert zu
\begin{equation*}
    \int I_C \; \symup{d} t = C \cdot U_a.
\end{equation*}
Über die Knotenregel, sowie dem Zusammenhang $U = R \cdot I$ ergibt sich für die Ausgangsspannung
\begin{equation*}
    U_a = - \frac{1}{RC} \int U_e (t) \symup{d} t
\end{equation*}
und damit für ein sinusförmiges Eingangssignal $U_e = U_0 \sin(\omega t)$
\begin{equation*}
    U_a = \frac{U_0}{\omega R C} \cos(\omega t).
\end{equation*}
Es ergibt sich also ein antiproportionaler Zusammenhang zwischen der Ausgangsspannung $U_a$
und der Frequenz $\omega$. Allgemein können auch alle anderen Signale integriert werden.
Bei Unstetigkeiten kann es dabei allerdings zum sogenannten Gibb'schen Phänomen kommen,
also zu Überschwingungen an den Unstetigkeiten. 

\subsection{Umkehr-Differentiator}
Werden nun Kondensator und Widerstand aus der vorherigen Schaltung vertauscht, ergibt sich 
ein Umkehr-Differentiator. Die Schaltung dazu ist in \autoref{fig:umkehrdiff}
dargestellt.
\begin{figure}[H]
    \centering
    \includegraphics[width=0.4\textwidth]{diff.png}
    \caption{Schaltskizze des Umkehr-Differenzierers \cite{anleitung}.}
    \label{fig:umkehrdiff}
\end{figure}
Mit der Knotenregel ergibt sich
\begin{equation*}
    C \frac{\symup{d} U_e(t)}{\symup{d} t} + \frac{U_a(t)}{R} = 0
\end{equation*}
und damit durch Differentiation und Umstellen
\begin{equation*}
    U_a = - R C \frac{\symup{d}U_e}{\symup{d} t}
\end{equation*}
und damit wieder für ein sinusförmiges Signal 
\begin{equation*}
    U_a = - \omega R C U_0 \cos(\omega t).
\end{equation*}
Die Ausgangsspannung ist also direkt proportional zur Frequenz des Eingangssignals.

\subsection{Nicht-invertierender Schmitt-Trigger}
Um den Schmitt-Trigger zu bauen, wie in \autoref{fig:schmitt} zu sehen, wird eine sogenannte
Mitkopplung auf den nicht-invertierenden Eingang über einen Widerstand, so erhält der
Operationsverstärker die Eigenschaft eines Schalters.
\begin{figure}[H]
    \centering
    \includegraphics[width=0.4\textwidth]{schmitt.png}
    \caption{Schaltskizze des nicht-invertierenden Schmitt-Triggers \cite{anleitung}.}
    \label{fig:schmitt}
\end{figure}
Dies bedeutet, dass der Operationsverstärker
bei einem Schwellwert $U_{S_{+}}$ schlagartig den Sättigungswert $U_B$ annimmt.
Die Schwellspannung ist dabei abhängig davon ob vor Veränderung des Eingangs eine 
positive oder negative Ausgangsspannung vorliegt. 
Die Schwellspannungen sind 
\begin{equation*}
    U_{S_{\pm}} = U_{\pm} \frac{R_1}{R_2}.
\end{equation*}

\subsection{Signalgenerator}
Durch Kombination von Schmitt-Trigger (\autoref{fig:schmitt}) und Integrator (\autoref{fig:umkehrint})
ist es möglich einen Signalgenerator zu bauen, wie er in \autoref{fig:signal} skizziert ist.
Der Schmitt-Trigger liefert eine konstante Spannung $U_B$ bis der Schwellwert $U_{S_{-}}$
unterschritten wird. Bei Verwenden eines sinusförmigen Signals gibt der Schmitt-Triggers
ein Rechtecksignal aus. Der Integrator integriert die Spannung zu einer Dreiecksspannung, die dort 
ebenfalls abgegriffen werden kann. Die Frequenz der Schwingung hat den Theoriewert
\begin{equation*}
    \nu_{\text{Dreieck}} = \frac{R_2}{4 C R_1 R_3}.
\end{equation*}
\begin{figure}[H]
    \centering
    \includegraphics[width=0.4\textwidth]{signalgenerator.png}
    \caption{Schaltskizze des Signalgenerators \cite{anleitung}.}
    \label{fig:signal}
\end{figure}

\subsection{Variierende Amplituden}
Durch das Hintereinanderschalten von zwei Integratoren und einem invertierenden Linearverstärker
kann eine gedämpfte harmonische Schwingung erzeugt werden. Die Schaltskizze ist in 
\autoref{fig:amplituden} dargestellt.
\begin{figure}[H]
    \centering
    \includegraphics[width=0.4\textwidth]{integrator.png}
    \caption{Schaltskizze für variierende Sinusamplituden \cite{anleitung}.}
    \label{fig:amplituden}
\end{figure}
Diese Schwingung lässt sich durch die Differentialgleichung
\begin{equation*}
    \frac{\symup{d^2} U_a}{\symup{d} t^2} - \frac{\eta}{10 R C} \frac{\symup{d} U_a}{\symup{d} t} + \frac{1}{R^2 C^2} U_a = 0
\end{equation*}
beschreiben, dessen Lösung 
\begin{equation*}
    U_a (t) = U_0 \exp \left( \frac{\eta}{10 R C} \right) \sin \left(frac{t}{R C} \right)
\end{equation*}
ist.
Dabei beschreibt $ -1 \leq \eta \leq 1 $ eine durch ein Potentiometer einstellbare Dämpfung.
Die Schwingungsdauer $T$ ist gegeben durch das sinus-Argument der Lösung der Differentialgleichung zu
\begin{equation*}
    T = 2 \pi R C
\end{equation*}
und die Abklingdauer beziehungsweise Zunahmedauer ist gegeben durch den Exponentialterm zu
\begin{equation*}
    \tau = \frac{20 R C}{\vert \eta \vert}.
\end{equation*}
