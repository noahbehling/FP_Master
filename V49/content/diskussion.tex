\section{Diskussion}
Die Messung der Spin-Gitter Relaxationszeit verlief erwartungsgemäß. Ein Vergleich zwischen gemessenen Wert und Theoriewert\cite{SpinGitter}
\begin{align*}
    T_{1, \text{gem}} &= (2,41 \pm 0,18) \, \symup{s} \\
    T_{1, \text{the}} &= (3,15 \pm 0,06) \, \symup{s},
\end{align*}
ergibt eine relative Abweichung von $23,5\%$. Dabei gilt es aber zu Beachten, dass der Theoriewert für eine Temperatur
von $T = 20\,\symup{°C}$ gilt, während die Messung bei $T = 22,4\,\symup{°C}$ stattfand. 
Dies ist ein Grund für die große relative Abweichung. Ein weiterer ist die Ungenauigkeit der Messung an sich.
Die Messwerte werden dabei manuell mit dem Cursor des Oszilloskops aufgenommen, was die Genauigkeit durch 
die grafische Auflösungsbegrenzung des Oszilloskops und durch menschliche Ungenauigkeit begrenzt.
Bei Betrachtung von \autoref{fig:T1} fällt zudem auf, dass im flachen Teil des Plots die Messwerte 
stark schwanken, was ein weiteres Indiz für die Ungenauigkeit der Messung mit Oszilloskop ist.
Eine Verbesserung der Genauigkeit der Messung ist zu dem durch die Aufnahme weiterer Messwerte
im ansteigenden Bereich des Graphen, also im Größenordnungsbereich von $1\,\symup{s}$, möglich, da 
hier vergleichsweise wenige Messwerte aufgenommen wurden, dieser Bereich der Grafik aber relevanter 
für den Fit ist.\\
Die Messung der Spin-Spin Relaxationszeit wurde mit der Meiboom-Gill Methode durchgeführt.
Die gefundenen Peaks in \autoref{fig:T2} folgen dem Verlauf, bis auf wenige Ausnahmen,
der angepassten Kurve sehr gut. Der gefundene Wert 
\begin{equation*}
    T_2 = (1,03 \pm 0,07) \, \symup{s}
\end{equation*}
weist daher auch nur einen geringe Unsicherheit auf.
Die Messung von $T_2$ wird deshalb als gut angenommen.\\
Der Messwert und der Theoriewert\cite{diff} des Diffusionskoeffizienten lauten 
\begin{align*}
    D_{\text{gem}} &= (2,18 \pm 0,05 ) \cdot 10^{-9} \, \frac{\symup{m^2}}{\symup{s}} \\
    D_{\text{the}} &= 2,0953\cdot 10^{-9} \, \frac{\symup{m^2}}{\symup{s}},
\end{align*}
bei jeweils $21,4 \,°\symup{C}$. Dadurch ergibt sich eine relative Abweichung von $4\%$.
Dies ist eine sehr geringe Abweichung, was für eine sehr gute Messung beim Diffusionskoeffizienten spricht
und das obwohl die $\tau^3$ in \autoref{fig:quanti} nur sehr grob nachgewiesen werden konnte.\\
Der bestimmte Wert für den Molekülradius wird mit dem Radius eines Wassermoleküls aus hexagonal dichtesten 
Kugelpackungen, sowie einem Weiteren Literaturwert\cite{radius} verglichen. Die Werte lauten
\begin{align*}
    r_{\text{gem}} &= (0,98 \pm 0,02) \, \symup{\mathring{A}}\\
    r_{\text{hdk}} &=1,74 \, \symup{\mathring{A}} \\
    r_{\text{Lit}} &= 1,69 \, \symup{\mathring{A}}. 
\end{align*}
Es fällt auf, dass $r_{\text{hdk}}$ nur um $3\%$ von $r_{\text{Lit}}$ abweicht, was darauf schließen lässt, 
dass das Modell der dichtesten Kugelpackungen die Realität gut annähert. Der gemessene Wert weicht 
um $42\%$ vom Literaturwert ab. Da der Diffusionskoeffizient nur eine geringe Abweichung vom 
Literaturwert aufweist, ist davon auszugehen, dass die Problematik in der Temperatur und der damit 
verbundenen Viskosität liegt. Einerseits spiegelt die Temperaturangabe nur eine Momentaufnahme im Experiment 
wider, andererseits war es nicht möglich, geeignete Werte für die Viskosität bei dieser Temperatur zu finden. \\
Insgesamt sind die Messungen aber dennoch als gut zu bewerten. Abweichungen sind entweder sehr gering oder 
durch die schwankende Temperaturen beziehungsweise schlechte Ablesmöglichkeiten zu erklären.
Im Versuch war leider keine Viskositätsmessung möglich; eine solche hätte möglicherweise das Ergebnis des
Molekülradius verbessert.

